%% Public domain image from
%% http://www.public-domain-image.com/objects/computer-chips/slides/six-computers-chips-circuits.html
\renewcommand\chapterillustration{cherry-tomatos.jpg}
\chapter{引言}
\label{引言}


我想给大家讲个故事。

\par 故事内容不是 Linus Torvalds 在1991年怎样写了 Linux 内核的第一个版本, 因为这些内容你可以在许多 Linux 书籍中读到。我也不想告诉你,更早之前,Richard Stallman 是如何开始 GNU 项目,设计了一个免费的类似 Unix 的操作系统。那也是一个很有意义的故事, 但大多数 Linux 书籍也讲到了它。

\par 我想告诉大家一个你如何才能夺回计算机管理权的故事。

\par 在20世纪70年代末,我刚开始和计算机打交道时,正进行着一场革命,那时的我还是一名大学生。 微处理器的发明,让你我这样的普通老百姓也有可能真正拥有一台计算机。今天, 人们很难想象,只有大企业和强大的政府才能够拥有计算机的世界,是怎样的一个世界。 让我说,你想不出多少来。

\par 今天,世界已经截然不同了。计算机遍布各个领域,从小手表到大型数据中心,及大小介于它们之间的每件东西。 除了随处可见的计算机之外,我们还有一个无处不在的连接所有计算机的网络。这已经开创了一个人们可以 自我营造和自由创作的奇妙的新时代,但在过去的二三十年里,一些事情仍然在发生着改变。一个大公司不断地把它的 管理权强加到世界上绝大多数的计算机上,并且决定你对计算机的操作权力。幸运地是,来自世界各地的人们, 正积极努力地做些事情来改变这种境况。通过编写自己的软件,他们一直在为维护电脑的管理权而战斗着。 他们建设着 Linux。

\par 一提到 Linux,许多人都会说到“自由”,但我不认为他们都知道“自由”的真正涵义。“自由”是一种权力, 它决定你的计算机能做什么,同时,只有知道计算机正在做什么你才能够拥有这种“自由”。 “自由”是指一台没有任何秘密的计算机,你可以从它那里了解一切,只要你用心的去寻找。

\section{ls 乐趣}

有充分的理由证明,ls 可能是用户最常使用的命令。通过它,我们可以知道目录的内容,以及各种各样重要文件和目录的 属性。正如我们所知道的,只简单的输入 ls 就能看到在当前目录下所包含的文件和子目录列表。

\begin{lstlisting}
[me@linuxbox ~]$ ls
Desktop Documents Music Pictures Publica Templates Videos 
\end{lstlisting}

\par 除了当前工作目录以外,也可以列出指定目录的内容,就像这样:

\begin{lstlisting}
me@linuxbox ~]$ ls /usr
bin games   kerberos    libexec  sbin   src
etc include lib         local    share  tmp 
\end{lstlisting}

\par 甚至可以列出多个指定目录的内容。在这个例子中,将会列出用户主目录(用字符“~”代表)和/usr 目录的内容:

\begin{lstlisting}
[me@linuxbox ~]$ ls ~ /usr
/home/me:
Desktop  Documents  Music  Pictures  Public  Templates  Videos

/usr:
bin  games      kerberos  libexec  sbin   src
etc  include    lib       local    share  tmp 
\end{lstlisting}

\par 我们也可以改变输出格式,来得到更多的细节:

\begin{lstlisting}
[me@linuxbox ~]$ ls -l
total 56
drwxrwxr-x 2  me  me  4096  2007-10-26  17:20  Desktop
drwxrwxr-x 2  me  me  4096  2007-10-26  17:20  Documents
drwxrwxr-x 2  me  me  4096  2007-10-26  17:20  Music
drwxrwxr-x 2  me  me  4096  2007-10-26  17:20  Pictures
drwxrwxr-x 2  me  me  4096  2007-10-26  17:20  Public
drwxrwxr-x 2  me  me  4096  2007-10-26  17:20  Templates
drwxrwxr-x 2  me  me  4096  2007-10-26  17:20  Videos
\end{lstlisting}

\par 使用 ls 命令的“-l”选项,则结果以长模式输出。


\subsection{选项和参数}
我们将学习一个非常重要的知识点,大多数命令是如何工作的。命令名经常会带有一个或多个用来更正命令行为的选项, 更进一步,选项后面会带有一个或多个参数,这些参数是命令作用的对象。所以大多数命令看起来像这样:

\begin{lstlisting}
command -options arguments
\end{lstlisting}

\par 大多数命令使用的选项,是由一个中划线加上一个字符组成,例如,“-l”,但是许多命令,包括来自于 GNU 项目的命令,也支持长选项,长选项由两个中划线加上一个字组成。当然, 许多命令也允许把多个短选项串在一起使用。下面这个例子,ls 命令有两个选项, “l” 选项产生长格式输出,“t”选项按文件修改时间的先后来排序。

\begin{lstlisting}
[me@linuxbox ~]$ ls -lt
\end{lstlisting}

\par 加上长选项 “–reverse”,则结果会以相反的顺序输出:

\begin{lstlisting}
[me@linuxbox ~]$ ls -lt --reverse
\end{lstlisting}

\par ls 命令有大量的选项。表3-1列出了最常使用的选项。

\begin{table}[ht!]
% increase table row spacing, adjust to taste
%\renewcommand{\arraystretch}{1.2}
\caption{ls 命令选项}
\label{table_example}
\centering
\begin{tabular}{p{1.5cm}p{3.5cm}p{10cm}}
%\begin{tabular}{c|c|c}
\hline
%\begin{tabular}{|p{0.18\textwidth}|p{0.36\textwidth}|p{0.36\textwidth}|}
 选项 & 长选项 & 描述 \\
\hline
 -a & --all & 列出所有文件,甚至包括文件名以圆点开头的隐藏文件。 \\
-d & --directory & 通常,如果指定了目录名,ls 命令会列出这个目录中的内容,而不是目录本身。 把这个选项与-l 选项结合使用,可以看到所指定目录的详细信息,而不是目录中的内容。\\
-F & --classify	& 这个选项会在每个所列出的名字后面加上一个指示符。例如,如果名字是 目录名,则会加上一个``/''字符。 \\
-h & --human-readable & 以长格式列出。以人们可读的格式,而不是以字节数来显示文件的大小。\\
-l & & 以长格式显示结果。\\
-r & --reverse & 以相反的顺序来显示结果。通常,ls 命令的输出结果按照字母升序排列。 \\
-S & & 命令输出结果按照文件大小来排序。\\
-t & & 按照修改时间来排序。\\
\hline
\end{tabular}
\end{table}


\subsection{深入研究长格式输出}
正如我们先前知道的,“-l”选项导致 ls 的输出结果以长格式输出。这种格式包含大量的有用信息。下面的例子目录来自 于 Ubuntu 系统:
(注:由于排版原因,删除了原文的的一部分.)
\begin{lstlisting}
-rw-r--r-- 1 root root 3576296 2007-04-03 11:05 Experience ubuntu.ogg
-rw-r--r-- 1 root root 1186219 2007-04-03 11:05 kubuntu-leaflet.png 
-rw-r--r-- 1 root root   47584 2007-04-03 11:05 logo-Edubuntu.png 
-rw-r--r-- 1 root root   44355 2007-04-03 11:05 logo-Kubuntu.png
-rw-r--r-- 1 root root   34391 2007-04-03 11:05 logo-Ubuntu.png
-rw-r--r-- 1 root root   32059 2007-04-03 11:05 oo-cd-cover.odf
\end{lstlisting}

\par 选一个文件,来看一下各个输出字段的含义(表3-2):

\begin{table}[ht!]
% increase table row spacing, adjust to taste
%\renewcommand{\arraystretch}{1.2}
\caption{ls 长格式列表的字段}
\label{table_example}
\centering
\begin{tabular}{p{4cm}p{11cm}}
%\begin{tabular}{c|c|c}
\hline
%\begin{tabular}{|p{0.18\textwidth}|p{0.36\textwidth}|p{0.36\textwidth}|}
 字段 & 含义 \\
 \hline 
 -rw-r--r--	& 对于文件的访问权限。第一个字符指明文件类型。在不同类型之间,开头的“-”说明是一个普通文件,“d”表明是一个目录。其后三个字符是文件所有者的访问权限,再其后的三个字符是文件所属组中成员的访问权限,最后三个字符是其他所 有人的访问权限。这个字段的完整含义将在第十章讨论。\\
1 & 文件的硬链接数目。参考随后讨论的关于链接的内容。 \\
root & 文件属主的用户名。 \\
root & 文件所属用户组的名字。 \\
32059 & 以字节数表示的文件大小。\\
2007-04-03 11:05 & 上次修改文件的时间和日期。 \\
oo-cd-cover.odf	& 文件名。 \\
\hline

\end{tabular}
\end{table}

\section{mkdir - 创建文件夹} % (fold)
\label{sec:mkdir_创建文件夹}
mkdir 命令是用来创建目录的。它这样工作:
\begin{lstlisting}
mkdir directory...
\end{lstlisting}
\par \textbf{注意表示法}: 在描述一个命令时(如上所示),当有三个圆点跟在一个命令的参数后面, 这意味着那个参数可以重复,就像这样:
\begin{lstlisting}
mkdir dir1
\end{lstlisting}
\par 会创建一个名为”dir1”的目录,而
\begin{lstlisting}
mkdir dir1 dir2 dir3
\end{lstlisting}
\par 会创建三个目录,名为``dir1'', ``dir2'', ``dir3''。


% section mkdir_创建文件夹 (end)

\section{用 less 浏览文件内容} % (fold)
\label{sec:用_less_浏览文件内容}

less 命令是一个用来浏览文本文件的程序。纵观 Linux 系统,有许多人类可读的文本文件。less 程序为我们检查文本文件 提供了方便。

\fboxrule=6pt \fboxsep=4pt
\begin{colorboxed}[boxcolor=lightgray,bgcolor=white]
\subsection{什么是``文本''}
在计算机中,有许多方法可以表达信息。所有的方法都涉及到,在信息与一些数字之间确立一种关系,而这些数字可以 用来表达信息。毕竟,计算机只能理解数字,这样所有的数据都被转换成数值表示法。

\par 有些数值表达法非常复杂(例如压缩的视频文件),而其它的就相当简单。最早也是最简单的一种表达法,叫做 ASCII 文本。ASCII(发音是"As-Key")是美国信息交换标准码的简称。这是一个简单的编码方法,它首先 被用在电传打字机上,用来实现键盘字符到数字的映射。

\par 文本是简单的字符与数字之间的一对一映射。它非常紧凑。五十个字符的文本翻译成五十个字节的数据。文本只是包含 简单的字符到数字的映射,理解这点很重要。它和一些文字处理器文档不一样,比如说由微软和 OpenOffice.org 文档 编辑器创建的文件。这些文件,和简单的 ASCII 文件形成鲜明对比,它们包含许多非文本元素,来描述它的结构和格式。 普通的 ASCII 文件,只包含字符本身,和一些基本的控制符,像制表符,回车符及换行符。纵观 Linux 系统,许多文件 以文本格式存储,也有许多 Linux 工具来处理文本文件。甚至 Windows 也承认这种文件格式的重要性。著名的 NOTEPAD.EXE 程序就是一个 ASCII 文本文件编辑器。
\end{colorboxed}

\par 为什么我们要查看文本文件呢? 因为许多包含系统设置的文件(叫做配置文件),是以文本格式存储的,阅读它们 可以更深入的了解系统是如何工作的。另外,许多系统所用到的实际程序(叫做脚本)也是以这种格式存储的。 在随后的章节里,我们将要学习怎样编辑文本文件,为的是修改系统设置,还要学习编写自己的脚本文件,但现在我们只是看看它们的内容而已。

\par less 命令是这样使用的:

\begin{lstlisting}
less filename
\end{lstlisting}

\par 一旦运行起来,less 程序允许你前后滚动文件。例如,要查看一个定义了系统中全部用户身份的文件,输入以下命令:
\begin{lstlisting}
[me@linuxbox ~]$ less /etc/passwd
\end{lstlisting}

\par 一旦 less 程序运行起来,我们就能浏览文件内容了。如果文件内容多于一页,那么我们可以上下滚动文件。按下“q”键, 退出 less 程序。

\par 下表列出了 less 程序最常使用的键盘命令。
\begin{table}[ht!]
% increase table row spacing, adjust to taste
%\renewcommand{\arraystretch}{1.2}
\caption{less 命令}
\label{table3}
\centering
\begin{tabular}{p{4cm}p{11cm}}
%\begin{tabular}{c|c|c}
\hline
%\begin{tabular}{|p{0.18\textwidth}|p{0.36\textwidth}|p{0.36\textwidth}|}
命令 & 行为\\

\hline
Page UP or b &	向后翻滚一页\\
Page Down or space & 向前翻动一页 \\
UP Arrow & 向前移动一行\\
Down Arrow &	向后移动一行\\
G	& 移动到最后一行\\
1G or g	& 移动到开头一行\\
/charaters	& 向前查找指定的字符串\\
n	& 向前查找下一个出现的字符串,这个字符串是之前所指定查找的\\
h	& 显示帮助屏幕\\
q	& 退出 less 程序\\
\hline
\end{tabular}
\end{table}

\fboxrule=6pt \fboxsep=4pt
\begin{colorboxed}[boxcolor=lightgray,bgcolor=white]
\subsection{less 就是 more(禅语:色即是空)}

ess 程序是早期 Unix 程序 more 的改进版。“less” 这个名字,对习语 “less is more” 开了个玩笑, 这个习语是现代主义建筑师和设计者的座右铭。

\par less 属于”页面调度器”程序类,这些程序允许通过页方式,在一页中轻松地浏览长长的文本文档。然而 more 程序只能向前分页浏览,而 less 程序允许前后分页浏览,它还有很多其它的特性。
\end{colorboxed}


% section 用_less_浏览文件内容 (end)

\section{更改当前工作目录} % (fold)
\label{sec:更改当前工作目录}
\par 要更改工作目录(此刻,我们站在树形迷宫里面),我们用 cd 命令。输入 cd, 然后输入你想要的工作目录的路径名,就能实现愿望。路径名就是沿着目录树的分支 到达想要的目录,期间所经过的路线。路径名可通过两种方式来指定,一个是绝对路径, 另一个是相对路径。首先处理绝对路径。

\subsection{绝对路径} % (fold)
\label{ssub:绝对路径}
绝对路径开始于根目录,紧跟着目录树的一个个分支,一直到达期望的目录或文件。 例如,你的系统中有一个目录,大多数系统程序都安装在这个目录下。这个目录的 路径名是 /usr/bin。它意味着从根目录(用开头的“/”表示)开始,有一个叫 “usr” 的 目录包含了目录 “bin”。
\begin{lstlisting}
[me@linuxbox ~]$ cd /usr/bin
[me@linuxbox bin]$ pwd
/usr/bin
[me@linuxbox bin]$ ls
...Listing of many, many files ...
\end{lstlisting}
\par 我们把工作目录转到 /usr/bin 目录下,里面装满了文件。注意 shell 提示符是怎样改变的。 为了方便,通常设置提示符自动显示工作目录名。
% subsection 绝对路径 (end)

\subsection{相对路径} % (fold)
\label{sub:相对路径}
绝对路径从根目录开始,直到它的目的地,而相对路径开始于工作目录。 一对特殊符号来表示相对位置,在文件系统树中。这对特殊符号是 ``.'' (点) 和 ``..'' (点点)。

\par 符号 ``.'' 指的是工作目录,``..'' 指的是工作目录的父目录。下面的例子说明怎样使用它。 再次更改工作目录到 /usr/bin:
\begin{lstlisting}
[me@linuxbox ~]$ cd /usr/bin
[me@linuxbox bin]$ pwd
/usr/bin
\end{lstlisting}

\par 好的,比方说更改工作目录到 /usr/bin 的父目录 /usr。可以通过两种方法来实现。或者使用绝对路径名:

\begin{lstlisting}
[me@linuxbox bin]$ cd /usr
[me@linuxbox usr]$ pwd
/usr
\end{lstlisting}

\par 或者, 使用相对路径:

\begin{lstlisting}
[me@linuxbox bin]$ cd ..
[me@linuxbox usr]$ pwd
/usr
\end{lstlisting}

\par 两种不同的方法,一样的结果。我们应该选哪一个呢? 输入量最少的那个。

\par 同样地,从目录/usr/到/usr/bin 也有两种途径。或者使用绝对路径:

\begin{lstlisting}
[me@linuxbox usr]$ cd /usr/bin
[me@linuxbox bin]$ pwd
/usr/bin
\end{lstlisting}

\par 或者,用相对路径:

\begin{lstlisting}
[me@linuxbox usr]$ cd ./bin
[me@linuxbox bin]$ pwd
/usr/bin
\end{lstlisting}

\par 有一件很重要的事,我必须指出来。在几乎所有的情况下,你可以省略”./”。它是隐含地。输入:
\begin{lstlisting}
[me@linuxbox usr]$ cd bin
\end{lstlisting}

\par 实现相同的效果,如果不指定一个文件的目录,那它的工作目录会被假定为当前工作目录。

% subsection 相对路径 (end)

\subsubsection{有用的快捷键} % (fold)
\label{ssub:有用的快捷键}
在表2-1中,列举出了一些快速改变当前工作目录的有效方法。


\begin{table}[ht!]
% increase table row spacing, adjust to taste
%\renewcommand{\arraystretch}{1.2}
\caption{cd 快捷键}
\label{table_example}
\centering
\begin{tabular}{c|c}
\hline
%\begin{tabular}{|p{0.18\textwidth}|p{0.36\textwidth}|p{0.36\textwidth}|}
 快捷键 & 运行结果 \\
\hline
  cd	& 更改工作目录到主目录。 \\
  cd -	& 更改工作目录到先前的工作目录。\\
cd ~user\_name	& 更改工作目录到用户主目录。例如, cd ~bob 会更改工作目录到用户“bob”的主目录。\\
\hline
\end{tabular}
\end{table}

% subsubsection 有用的快捷键 (end)
\fboxrule=6pt \fboxsep=4pt
\begin{colorboxed}[boxcolor=lightgray,bgcolor=white]
\subsubsection{关于文件名的重要规则}
\begin{enumerate}
	\item 以 ``.'' 字符开头的文件名是隐藏文件。这仅表示,ls 命令不能列出它们, 除非使用 ls -a 命令。当你创建帐号后,几个配置帐号的隐藏文件被放置在 你的主目录下。稍后,我们会仔细研究一些隐藏文件,来定制你的系统环境。 另外,一些应用程序也会把它们的配置文件以隐藏文件的形式放在你的主目录下面。
	\item 文件名和命令名是大小写敏感的。文件名 “File1” 和 “file1” 是指两个不同的文件名。
	\item Linux 没有“文件扩展名”的概念,不像其它一些系统。可以用你喜欢的任何名字 来给文件起名。文件内容或用途由其它方法来决定。虽然类似 Unix 的操作系统, 不用文件扩展名来决定文件的内容或用途,但是应用程序会。
	\item 虽然 Linux 支持长文件名,文件名可能包含空格,标点符号,但标点符号仅限 使用 ``.'',``-'',下划线。最重要的是,不要在文件名中使用空格。如果你想表示词与 词间的空格,用下划线字符来代替。过些时候,你会感激自己这样做。
\end{enumerate}
\end{colorboxed}
% section 更改当前工作目录 (end)

\section{怎样阅读这本书} % (fold)
\label{sec:怎样阅读这本书}

从头到尾的阅读。它并不是一本技术参考手册,实际上它更像一本故事书,有开头,过程,结尾。

\subsection{前提条件} % (fold)
\label{ssub:_前提条件}
为了使用这本书,你需要安装 Linux 操作系统。你可以通过两种方式,来完成安装。
\begin{itemize}
\item 1. 在一台(不是很新)的电脑上安装 Linux。你选择哪个 Linux 发行版安装,是无关紧要的事。 虽然大多数人一开始选择安装 Ubuntu, Fedora, 或者 OpenSUSE。如果你拿不定主意,那就先试试 Ubuntu。 由于主机硬件配置不同,安装 Linux 时,你可能不费吹灰之力就装上了,也可能费了九牛二虎之力还装不上。 所以我建议,一台使用了几年的台式机,至少要有256M 的内存,6G 的硬盘可用空间。尽可能避免使用 笔记本电脑和无线网络,在 Linux 环境下,它们经常不能工作。

\item 2. 使用“Live CD.” 许多 Linux 发行版都自带一个比较酷的功能,你可以直接从系统安装盘 CDROM 中运行 Linux, 而不必安装 Linux。开机进入 BIOS 设置界面,更改引导项,设置为“从 CDROM 启动”。
\end{itemize}

\par 不管你怎样安装 Linux,为了练习书中介绍的知识,你需要有超级用户(管理员)权限。

\par 当你在自己的电脑上安装了 Linux 系统之后,就开始一边阅读本书,一边练习吧。本书大部分内容 都可以自己动手练习,坐下来,敲入命令,体验一下吧。



%\lemmabox{

\fboxrule=6pt \fboxsep=4pt
\begin{colorboxed}[boxcolor=lightgray,bgcolor=white]
 \subsubsection{为什么我不叫它“GNU/Linux”}

\par 在某些领域,把 Linux 操作系统称为“GNU/Linux 操作系统.”是比较明智的做法。但“Linux”的问题在于, 没有一个完全正确的方式能为它命名,因为它是由许许多多,分布在世界各地的贡献者们,合作开发而成的。 从技术层面讲,Linux 只是操作系统的内核名字,没别的含义。当然内核非常重要,有了内核, 操作系统才能运行起来,但它并不能构成一个完整的操作系统。

\par Richard Stallman 是一个天才的哲学家,自由软件运动创始人,自由软件基金会创办者,他创建了 GNU 工程, 编写了第一版 GNU C 编译器(gcc),创立了 GNU 通用公共协议(the GPL)等等。 他坚持把 Linux 称为“GNU/Linux”,为的是准确地反映 GNU 工程对 Linux 操作系统的贡献。 然而,GNU 项目早于 Linux 内核,而 GNU 项目的贡献得到了极高的赞誉,再把 GNU 用在 Linux 名字里, 这对其他每个为 Linux 的发展做出重大贡献的程序员来说,就不公平了。

\par 在目前流行的用法中,“Linux”指的是内核以及在一个典型的 Linux 发行版中所包含的所有免费及开源软件; 也就是说,整个 Linux 生态系统,不只有 GNU 项目软件。在操作系统商界,好像喜欢使用单个词的名字, 比如说 DOS, Windows, MacOS, Solaris, Irix, AIX. 所以我选择用流行的命名规则。然而, 如果你喜欢用“GNU/Linux”,当你读这本书时,可以搜索并代替“Linux”。我不介意。
\end{colorboxed}


% subsection _前提条件 (end)
% section 怎样阅读这本书 (end)

\section{拓展阅读} % (fold)
\label{sec:拓展阅读}

\begin{itemize}
	\item Wikipedia 网站上有些介绍本章提到的名人的文章,以下是链接地址:\\
	\url{http://en.wikipedia.org/wiki/Linux_Torvalds}\\
	\url{http://en.wikipedia.org/wiki/Richard_Stallman}
	\item 介绍自由软件基金会及 GNU 项目的网站和文章:\\
	\url{http://en.wikipedia.org/wiki/Free_Software_Foundation}\\
	\url{http://www.fsf.org}\\
	\url{http://www.gnu.org}
	\item Richard Stallman 用了大量的文字来叙述“GNU/Linux”的命名问题,可以浏览以下网页:\\
	\url{http://www.gnu.org/gnu/why-gnu-linux.html}\\
	\url{http://www.gnu.org/gnu/why-gnu-linux.html}\\
	\url{http://www.gnu.org/gnu/gnu-linux-faq.html#tools}
\end{itemize}
% section 拓展阅读 (end)