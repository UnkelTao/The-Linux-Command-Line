\section{ls 乐趣}

有充分的理由证明,ls 可能是用户最常使用的命令。通过它,我们可以知道目录的内容,以及各种各样重要文件和目录的 属性。正如我们所知道的,只简单的输入 ls 就能看到在当前目录下所包含的文件和子目录列表。

\begin{lstlisting}
[me@linuxbox ~]$ ls
Desktop Documents Music Pictures Publica Templates Videos 
\end{lstlisting}

\par 除了当前工作目录以外,也可以列出指定目录的内容,就像这样:

\begin{lstlisting}
me@linuxbox ~]$ ls /usr
bin games   kerberos    libexec  sbin   src
etc include lib         local    share  tmp 
\end{lstlisting}

\par 甚至可以列出多个指定目录的内容。在这个例子中,将会列出用户主目录(用字符“~”代表)和/usr 目录的内容:

\begin{lstlisting}
[me@linuxbox ~]$ ls ~ /usr
/home/me:
Desktop  Documents  Music  Pictures  Public  Templates  Videos

/usr:
bin  games      kerberos  libexec  sbin   src
etc  include    lib       local    share  tmp 
\end{lstlisting}

\par 我们也可以改变输出格式,来得到更多的细节:

\begin{lstlisting}
[me@linuxbox ~]$ ls -l
total 56
drwxrwxr-x 2  me  me  4096  2007-10-26  17:20  Desktop
drwxrwxr-x 2  me  me  4096  2007-10-26  17:20  Documents
drwxrwxr-x 2  me  me  4096  2007-10-26  17:20  Music
drwxrwxr-x 2  me  me  4096  2007-10-26  17:20  Pictures
drwxrwxr-x 2  me  me  4096  2007-10-26  17:20  Public
drwxrwxr-x 2  me  me  4096  2007-10-26  17:20  Templates
drwxrwxr-x 2  me  me  4096  2007-10-26  17:20  Videos
\end{lstlisting}

\par 使用 ls 命令的“-l”选项,则结果以长模式输出。


\subsection{选项和参数}
我们将学习一个非常重要的知识点,大多数命令是如何工作的。命令名经常会带有一个或多个用来更正命令行为的选项, 更进一步,选项后面会带有一个或多个参数,这些参数是命令作用的对象。所以大多数命令看起来像这样:

\begin{lstlisting}
command -options arguments
\end{lstlisting}

\par 大多数命令使用的选项,是由一个中划线加上一个字符组成,例如,“-l”,但是许多命令,包括来自于 GNU 项目的命令,也支持长选项,长选项由两个中划线加上一个字组成。当然, 许多命令也允许把多个短选项串在一起使用。下面这个例子,ls 命令有两个选项, “l” 选项产生长格式输出,“t”选项按文件修改时间的先后来排序。

\begin{lstlisting}
[me@linuxbox ~]$ ls -lt
\end{lstlisting}

\par 加上长选项 “–reverse”,则结果会以相反的顺序输出:

\begin{lstlisting}
[me@linuxbox ~]$ ls -lt --reverse
\end{lstlisting}

\par ls 命令有大量的选项。表3-1列出了最常使用的选项。

\begin{table}[ht!]
% increase table row spacing, adjust to taste
%\renewcommand{\arraystretch}{1.2}
\caption{ls 命令选项}
\label{table2}
\centering
\begin{tabular}{p{1.5cm}p{3.5cm}p{10cm}}
%\begin{tabular}{c|c|c}
\hline
%\begin{tabular}{|p{0.18\textwidth}|p{0.36\textwidth}|p{0.36\textwidth}|}
 选项 & 长选项 & 描述 \\
\hline
 -a & --all & 列出所有文件,甚至包括文件名以圆点开头的隐藏文件。 \\
-d & --directory & 通常,如果指定了目录名,ls 命令会列出这个目录中的内容,而不是目录本身。 把这个选项与-l 选项结合使用,可以看到所指定目录的详细信息,而不是目录中的内容。\\
-F & --classify	& 这个选项会在每个所列出的名字后面加上一个指示符。例如,如果名字是 目录名,则会加上一个``/''字符。 \\
-h & --human-readable & 以长格式列出。以人们可读的格式,而不是以字节数来显示文件的大小。\\
-l & & 以长格式显示结果。\\
-r & --reverse & 以相反的顺序来显示结果。通常,ls 命令的输出结果按照字母升序排列。 \\
-S & & 命令输出结果按照文件大小来排序。\\
-t & & 按照修改时间来排序。\\
\hline
\end{tabular}
\end{table}


\subsection{深入研究长格式输出}
正如我们先前知道的,“-l”选项导致 ls 的输出结果以长格式输出。这种格式包含大量的有用信息。下面的例子目录来自 于 Ubuntu 系统:
(注:由于排版原因,删除了原文的的一部分.)
\begin{lstlisting}
-rw-r--r-- 1 root root 3576296 2007-04-03 11:05 Experience ubuntu.ogg
-rw-r--r-- 1 root root 1186219 2007-04-03 11:05 kubuntu-leaflet.png 
-rw-r--r-- 1 root root   47584 2007-04-03 11:05 logo-Edubuntu.png 
-rw-r--r-- 1 root root   44355 2007-04-03 11:05 logo-Kubuntu.png
-rw-r--r-- 1 root root   34391 2007-04-03 11:05 logo-Ubuntu.png
-rw-r--r-- 1 root root   32059 2007-04-03 11:05 oo-cd-cover.odf
\end{lstlisting}

\par 选一个文件,来看一下各个输出字段的含义(表3-2):

\begin{table}[ht!]
% increase table row spacing, adjust to taste
%\renewcommand{\arraystretch}{1.2}
\caption{ls 长格式列表的字段}
\label{table_example}
\centering
\begin{tabular}{p{4cm}p{11cm}}
%\begin{tabular}{c|c|c}
\hline
%\begin{tabular}{|p{0.18\textwidth}|p{0.36\textwidth}|p{0.36\textwidth}|}
 字段 & 含义 \\
 \hline 
 -rw-r--r--	& 对于文件的访问权限。第一个字符指明文件类型。在不同类型之间,开头的“-”说明是一个普通文件,“d”表明是一个目录。其后三个字符是文件所有者的访问权限,再其后的三个字符是文件所属组中成员的访问权限,最后三个字符是其他所 有人的访问权限。这个字段的完整含义将在第十章讨论。\\
1 & 文件的硬链接数目。参考随后讨论的关于链接的内容。 \\
root & 文件属主的用户名。 \\
root & 文件所属用户组的名字。 \\
32059 & 以字节数表示的文件大小。\\
2007-04-03 11:05 & 上次修改文件的时间和日期。 \\
oo-cd-cover.odf	& 文件名。 \\
\hline

\end{tabular}
\end{table}