\section{旅行指南} % (fold)
\label{sec:旅行指南}

Linux 系统中,文件系统布局与类似 Unix 系统的文件布局很相似。实际上,一个已经发布的标准, 叫做 Linux 文件系统层次标准,详细说明了这种设计模式。不是所有Linux发行版都根据这个标准,但 大多数都是。

\par 下一步,我们将在文件系统中游玩,来了解 Linux 系统的工作原理。这会给你一个温习跳转命令的机会。 我们会发现很多有趣的文件都是普通的可读文本。将开始旅行,做做以下练习:
\begin{enumerate}
	\item cd 到给定目录
	\item 列出目录内容 ls -l
	\item 如果看到一个有趣的文件,用 file 命令确定文件内容
	\item 如果文件看起来像文本,试着用 less 命令浏览它
\end{enumerate}
\fboxrule=3pt \fboxsep=2pt
\begin{colorboxed}[boxcolor=lightgray,bgcolor=white]
\textbf{记得复制和粘贴技巧!}如果你正在使用鼠标,双击文件名,来复制它,然后按下鼠标中键,粘贴文件名到命令行中。
\end{colorboxed}

\par 在系统中游玩时,不要害怕粘花惹草。普通用户是很难把东西弄乱的。那是系统管理员的工作! 如果一个命令抱怨一些事情,不要管它,尽管去玩别的东西。花一些时间四处走走。 系统是我们自己的,尽情地探究吧。记住在 Linux 中,没有秘密存在! 表3-4仅仅列出了一些我们可以浏览的目录。闲暇时试试看!


%跨页表格

\begin{center} \tablecaption{Linux 系统中的目录 \label{tab:super}}
\tablefirsthead{%
\rowcolor[gray]{0.8}
\multicolumn{1}{l}{\textbf{目录}} & 
\multicolumn{1}{c}{\textbf{注解}} \\ }
\tablehead{\multicolumn{2}{c}{%
\small 表 \ref{tab:super} (续) } \\
\rowcolor[gray]{0.8}
\multicolumn{1}{l}{\textbf{目录}} & 
\multicolumn{1}{c}{\textbf{注解}} \\}
\tabletail{\bottomrule
\multicolumn{2}{c}{\small 接下页} \\
}
\tablelasttail{\bottomrule}

\begin{supertabular}{p{2.cm}p{13cm}}

/	& 根目录,万物起源。\\
\midrule
/bin & 包含系统启动和运行所必须的二进制程序。\\
\midrule
/boot & 
包含 Linux 内核,最初的 RMA 磁盘映像(系统启动时,由驱动程序所需),和 启动加载程序。
有趣的文件:
\begin{itemize} 
	\item  /boot/grub/grub.conf or menu.lst, 被用来配置启动加载程序。
	\item /boot/vmlinuz,Linux 内核。
\end{itemize} \\
\midrule
/dev & 这是一个包含设备结点的特殊目录。“一切都是文件”,也使用于设备。 在这个目录里,内核维护着它支持的设备。\\
\midrule /etc & 	
这个目录包含所有系统层面的配置文件。它也包含一系列的 shell 脚本, 在系统启动时,这些脚本会运行每个系统服务。这个目录中的任何文件应该是可读的文本文件。
有意思的文件:虽然/etc 目录中的任何文件都有趣,但这里只列出了一些我一直喜欢的文件:
\begin{itemize}
	\item /etc/crontab, 定义自动运行的任务。
	\item /etc/fstab,包含存储设备的列表,以及与他们相关的挂载点。
	\item /etc/passwd,包含用户帐号列表。
\end{itemize} \\
\midrule /home & 在通常的配置环境下,系统会在/home 下,给每个用户分配一个目录。普通只能 在他们自己的目录下创建文件。这个限制保护系统免受错误的用户活动破坏。\\
\midrule /lib & 包含核心系统程序所需的库文件。这些文件与 Windows 中的动态链接库相似。 \\
\midrule/lost+found	& 每个使用 Linux 文件系统的格式化分区或设备,例如 ext3文件系统, 都会有这个目录。当部分恢复一个损坏的文件系统时,会用到这个目录。除非文件系统 真正的损坏了,那么这个目录会是个空目录。 \\
\midrule/media & 在现在的 Linux 系统中,/media 目录会包含可移除媒体设备的挂载点, 例如 USB 驱动器,CD-ROMs 等等。这些设备连接到计算机之后,会自动地挂载到这个目录结点下。 \\
\midrule/mnt & 在早些的 Linux 系统中,/mnt 目录包含可移除设备的挂载点。\\
\midrule /opt & 这个/opt 目录被用来安装“可选的”软件。这个主要用来存储可能 安装在系统中的商业软件产品。\\
\midrule/proc & 这个/proc 目录很特殊。从存储在硬盘上的文件的意义上说,它不是真正的文件系统。 反而,它是一个由 Linux 内核维护的虚拟文件系统。它所包含的文件是内核的窥视孔。这些文件是可读的, 它们会告诉你内核是怎样监管计算机的。\\
\midrule /root & root 帐户的主目录。\\
\midrule /sbin & 这个目录包含“系统”二进制文件。它们是完成重大系统任务的程序,通常为超级用户保留。\\
\midrule /tmp & 这个/tmp 目录,是用来存储由各种程序创建的临时文件的地方。一些配置,导致系统每次 重新启动时,都会清空这个目录。\\
\midrule /usr & 在 Linux 系统中,/usr 目录可能是最大的一个。它包含普通用户所需要的所有程序和文件。\\
\midrule /usr/bin & /usr/bin 目录包含系统安装的可执行程序。通常,这个目录会包含许多程序。\\
\midrule /usr/lib & 包含由/usr/bin 目录中的程序所用的共享库。\\
\midrule /usr/local & 这个/usr/local 目录,是非系统发行版自带,却打算让系统使用的程序的安装目录。 通常,由源码编译的程序会安装在/usr/local/bin 目录下。新安装的 Linux 系统中,会存在这个目录, 但却是空目录,直到系统管理员放些东西到它里面。\\
\midrule /usr/sbin & 包含许多系统管理程序。\\
\midrule /usr/share & /usr/share 目录包含许多由/usr/bin 目录中的程序使用的共享数据。 其中包括像默认的配置文件,图标,桌面背景,音频文件等等。\\
\midrule /usr/share/doc & 大多数安装在系统中的软件包会包含一些文档。在/usr/share/doc 目录下, 我们可以找到按照软件包分类的文档。\\
\midrule /var & 除了/tmp 和/home 目录之外,相对来说,目前我们看到的目录是静态的,这是说, 它们的内容不会改变。/var 目录是可能需要改动的文件存储的地方。各种数据库,假脱机文件, 用户邮件等等,都驻扎在这里。\\
\midrule /var/log & 这个/var/log 目录包含日志文件,各种系统活动的记录。这些文件非常重要,并且 应该时时监测它们。其中最重要的一个文件是/var/log/messages。注意,为了系统安全,在一些系统中, 你必须是超级用户才能查看这些日志文件。\\

\end{supertabular}
\end{center}




% section 旅行指南 (end)

