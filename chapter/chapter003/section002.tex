\section{确定文件类型} % (fold)
\label{sec:确定文件类型}
随着探究操作系统的进行,知道文件包含的内容是很有用的。我们将用 file 命令来确定文件的类型。我们之前讨论过, 在 Linux 统中,并不要求文件名来反映文件的内容。然而,一个类似 “picture.jpg” 的文件名,我们会期望它包含 JPEG 压缩图像,但 Linux 却不这样要求它。可以这样调用 file 命令:

\begin{lstlisting}
file filename
\end{lstlisting}

\par 当调用 file 命令后,file 命令会打印出文件内容的简单描述。例如:

\begin{lstlisting}
[me@linuxbox ~]$ file picture.jpg
picture.jpg: JPEG image data, JFIF standard 1.01
\end{lstlisting}

\par 有许多类型的文件。事实上,在类似于 Unix 操作系统中比如说 Linux,有个普遍的观念就是“任何东西都是一个文件”。 随着课程的进行,我们将会明白这句话的真谛。

\par 虽然系统中许多文件格式是熟悉的,例如 MP3和 JPEG 文件,但也有一些文件格式比较含蓄,极少数文件相当陌生。


% section 确定文件类型 (end)