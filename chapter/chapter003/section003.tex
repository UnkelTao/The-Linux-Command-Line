\section{用 less 浏览文件内容} % (fold)
\label{sec:用_less_浏览文件内容}

less 命令是一个用来浏览文本文件的程序。纵观 Linux 系统,有许多人类可读的文本文件。less 程序为我们检查文本文件 提供了方便。

\fboxrule=6pt \fboxsep=4pt
\begin{colorboxed}[boxcolor=lightgray,bgcolor=white]
\subsection{什么是``文本''}
在计算机中,有许多方法可以表达信息。所有的方法都涉及到,在信息与一些数字之间确立一种关系,而这些数字可以 用来表达信息。毕竟,计算机只能理解数字,这样所有的数据都被转换成数值表示法。

\par 有些数值表达法非常复杂(例如压缩的视频文件),而其它的就相当简单。最早也是最简单的一种表达法,叫做 ASCII 文本。ASCII(发音是"As-Key")是美国信息交换标准码的简称。这是一个简单的编码方法,它首先 被用在电传打字机上,用来实现键盘字符到数字的映射。

\par 文本是简单的字符与数字之间的一对一映射。它非常紧凑。五十个字符的文本翻译成五十个字节的数据。文本只是包含 简单的字符到数字的映射,理解这点很重要。它和一些文字处理器文档不一样,比如说由微软和 OpenOffice.org 文档 编辑器创建的文件。这些文件,和简单的 ASCII 文件形成鲜明对比,它们包含许多非文本元素,来描述它的结构和格式。 普通的 ASCII 文件,只包含字符本身,和一些基本的控制符,像制表符,回车符及换行符。纵观 Linux 系统,许多文件 以文本格式存储,也有许多 Linux 工具来处理文本文件。甚至 Windows 也承认这种文件格式的重要性。著名的 NOTEPAD.EXE 程序就是一个 ASCII 文本文件编辑器。
\end{colorboxed}

\par 为什么我们要查看文本文件呢? 因为许多包含系统设置的文件(叫做配置文件),是以文本格式存储的,阅读它们 可以更深入的了解系统是如何工作的。另外,许多系统所用到的实际程序(叫做脚本)也是以这种格式存储的。 在随后的章节里,我们将要学习怎样编辑文本文件,为的是修改系统设置,还要学习编写自己的脚本文件,但现在我们只是看看它们的内容而已。

\par less 命令是这样使用的:

\begin{lstlisting}
less filename
\end{lstlisting}

\par 一旦运行起来,less 程序允许你前后滚动文件。例如,要查看一个定义了系统中全部用户身份的文件,输入以下命令:
\begin{lstlisting}
[me@linuxbox ~]$ less /etc/passwd
\end{lstlisting}

\par 一旦 less 程序运行起来,我们就能浏览文件内容了。如果文件内容多于一页,那么我们可以上下滚动文件。按下“q”键, 退出 less 程序。

\par 下表列出了 less 程序最常使用的键盘命令。
\begin{table}[ht!]
% increase table row spacing, adjust to taste
%\renewcommand{\arraystretch}{1.2}
\caption{less 命令}
\label{table3}
\centering
\begin{tabular}{p{4cm}p{11cm}}
%\begin{tabular}{c|c|c}
\hline
%\begin{tabular}{|p{0.18\textwidth}|p{0.36\textwidth}|p{0.36\textwidth}|}
命令 & 行为\\

\hline
Page UP or b &	向后翻滚一页\\
Page Down or space & 向前翻动一页 \\
UP Arrow & 向前移动一行\\
Down Arrow &	向后移动一行\\
G	& 移动到最后一行\\
1G or g	& 移动到开头一行\\
/charaters	& 向前查找指定的字符串\\
n	& 向前查找下一个出现的字符串,这个字符串是之前所指定查找的\\
h	& 显示帮助屏幕\\
q	& 退出 less 程序\\
\hline
\end{tabular}
\end{table}

\fboxrule=6pt \fboxsep=4pt
\begin{colorboxed}[boxcolor=lightgray,bgcolor=white]
\subsection{less 就是 more(禅语:色即是空)}

ess 程序是早期 Unix 程序 more 的改进版。“less” 这个名字,对习语 “less is more” 开了个玩笑, 这个习语是现代主义建筑师和设计者的座右铭。

\par less 属于”页面调度器”程序类,这些程序允许通过页方式,在一页中轻松地浏览长长的文本文档。然而 more 程序只能向前分页浏览,而 less 程序允许前后分页浏览,它还有很多其它的特性。
\end{colorboxed}


% section 用_less_浏览文件内容 (end)