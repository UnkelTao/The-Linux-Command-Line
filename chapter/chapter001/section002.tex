\section{第一次按键} % (fold)
\label{sec:第一次按键}

好,开始吧。启动终端仿真器!一旦它运行起来,我们应该能够看到一行类似下面文字的输出:\\
\\
\begin{lstlisting}
[me@linuxbox ~]$
\end{lstlisting}

\par 这叫做 shell 提示符,当 shell 准备好了去接受输入时,它就会出现。然而, 它可能会以各种各样的面孔显示,这则取决于不同的 Linux 发行版, 它通常包括你的用户名\@主机名,紧接着当前工作目录(关于它的知识有点多)和一个美元符号。

\par 如果提示符的最后一个字符是``\verb"#"'', 而不是``\textdollar'', 那么这个终端会话就有超级用户权限。 这意味着,我们或者是以根用户的身份登录,或者是我们选择的终端仿真器提供超级用户(管理员)权限。

\par 假定到目前为止,所有事情都进行顺利,那我们试着打字吧。在提示符下敲入 一些乱七八糟的无用数据,如下所示:
\begin{lstlisting}
[me@linuxbox ~]$ kaekfjaeifj
\end{lstlisting}

\par 因为这个命令没有任何意义,所以 shell 会提示错误信息,并让我们再试一下:
\begin{lstlisting}
bash: kaekfjaeifj: command not found
[me@linuxbox ~]$
\end{lstlisting}

\subsection{命令历史}
如果按下上箭头按键,我们会看到刚才输入的命令“kaekfjaeifj”重新出现在提示符之后。 这就叫做命令历史。许多 Linux 发行版默认保存最后输入的500个命令。 按下下箭头按键,先前输入的命令就消失了。

\subsection{移动光标}
可借助上箭头按键,来回想起上次输入的命令。现在试着使用左右箭头按键。 看一下怎样把光标定位到命令行的任意位置?通过使用箭头按键,使编辑命令变得轻松些。

\fboxrule=6pt \fboxsep=4pt
\begin{colorboxed}[boxcolor=lightgray,bgcolor=white]
\subsubsection{关于鼠标和光标}
虽然,shell 是和键盘打交道的,但你也可以在终端仿真器里使用鼠标。X 窗口系统 (使 GUI 工作的底层引擎)内建了一种机制,支持快速拷贝和粘贴技巧。 如果你想高亮一些文本,可以按下鼠标左键,沿着文本拖动鼠标(或者双击一个单词), 那么这些高亮的文本就被拷贝到了一个由 X 管理的缓冲区里面。然后按下鼠标中键, 这些文本就被粘贴到光标所在的位置。试试看。

\par 注意: 不要受诱惑在一个终端窗口里,使用 Ctrl-c 和 Ctrl-v 快捷键,来执行拷贝和粘贴操作。 它们不起作用。对于 shell 来说,这些控制代码有着不同的含义,它们被赋值,早于 Microsoft Windows 许多年。

\par 你的图形桌面环境(像 KDE 或 GNOME),努力想和 Windows 一样,可能会把它的聚焦策略 设置成“单击聚焦”。这意味着,为了让窗口聚焦(变得活跃)你需要单击它。 这与“聚焦跟随着鼠标”的传统 X 行为相反,传统 X 行为是指只要把鼠标移动到一个窗口的上方, 这个窗口就成为活动窗口。这个窗口不会成为前端窗口,直到你单击它,但它能接受输入。 设置聚焦策略为“聚焦跟随着鼠标”,可以使拷贝和粘贴技巧更有益。尝试一下。 给它一个机会,我想你会喜欢上它的。在窗口管理器的配置程序中,你会找到这个设置。

\end{colorboxed}


% section 第一次按键 (end)