%% Invoke fancy unnumbered chapter style
%% for the table of contents
\chapterstyle{FancyUnnumberedChap}

\tableofcontents*
%% Main matter starts here; resets page-numberings to arabic numeral 1

%% Invoke the FancyChap chapter style
\chapterstyle{FancyChap}

\lstset{
	    %numbers=left, %设置行号位置
        %numberstyle=\tiny, %设置行号大小
        keywordstyle=\color{blue}, %设置关键字颜色
        commentstyle=\color[cmyk]{1,0,1,0}, %设置注释颜色
        frame=single, %设置边框格式
        escapeinside=``, %逃逸字符(1左面的键),用于显示中文
        breaklines, %自动折行
        extendedchars=false, %解决代码跨页时,章节标题,页眉等汉字不显示的问题
        xleftmargin=2em,xrightmargin=2em, aboveskip=1em, %设置边距
        tabsize=4, %设置tab空格数
        showspaces=false %不显示空格
       } %源代码风格

\XeTeXlinebreaklocale "zh"

% \renewcommand{\chaptername}{第\CJKnumber{\thechapter}章}
% \newcommand{\sectionname}{节}
\renewcommand{\figurename}{图}
\renewcommand{\tablename}{表}
% \renewcommand{\bibname}{参考文献}
% \renewcommand{\contentsname}{目~录}
% \renewcommand{\listfigurename}{图~目~录}
% \renewcommand{\listtablename}{表~目~录}
% \renewcommand{\indexname}{索~引}
% \renewcommand{\abstractname}{\Large{摘~要}}
% \newcommand{\keywords}[1]{\\ \\ \textbf{关~键~词}:#1}
% \titleformat{\chapter}[block]{\center\Large\bf}{\chaptername}{20pt}{}
% \titleformat{\section}[block]{\large\bf}{\thesection}{10pt}{}